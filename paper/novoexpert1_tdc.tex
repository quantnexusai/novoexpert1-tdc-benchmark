\documentclass[11pt]{article}

% Packages
\usepackage[utf8]{inputenc}
\usepackage[margin=1in]{geometry}
\usepackage{amsmath,amssymb}
\usepackage{graphicx}
\usepackage{booktabs}
\usepackage{hyperref}
\usepackage{natbib}
\usepackage{authblk}

% Title
\title{NovoExpert-1: State-of-the-Art CYP2D6 Prediction via Message-Passing Neural Networks on the TDC ADMET Benchmark}

\author[1]{Ari Harrison}
\affil[1]{NovoQuantNexus}

\date{February 2026}

\begin{document}

\maketitle

\begin{abstract}
We present NovoExpert-1, a suite of Chemprop-based message-passing neural networks for ADMET property prediction. Evaluated on the Therapeutics Data Commons (TDC) standardized benchmark, NovoExpert-1 achieves state-of-the-art performance on CYP2D6 metabolism prediction with an AUROC of 0.864, exceeding the prior best result of 0.750 by 11.4 percentage points. We also achieve near state-of-the-art results on CYP3A4 (0.890 vs 0.900) and CYP2C9 (0.878 vs 0.900). Our models use the directed message-passing neural network (D-MPNN) architecture with standardized hyperparameters, demonstrating that careful application of established methods can yield significant improvements on clinically relevant endpoints. Code and trained models are available at \url{https://github.com/quantnexusai/novoexpert1-tdc-benchmark}.
\end{abstract}

\section{Introduction}

Accurate prediction of absorption, distribution, metabolism, excretion, and toxicity (ADMET) properties is critical for successful drug discovery \citep{kola2004can}. Among metabolic enzymes, the cytochrome P450 (CYP) family is responsible for metabolizing approximately 75\% of clinically used drugs \citep{guengerich2008cytochrome}. CYP2D6 alone accounts for the metabolism of 25\% of marketed pharmaceuticals, including antidepressants, antipsychotics, and opioids \citep{zanger2013cytochrome}.

The Therapeutics Data Commons (TDC) provides standardized benchmarks for evaluating machine learning models on drug discovery tasks \citep{huang2021therapeutics}. The TDC ADMET benchmark group includes 22 endpoints with fixed train/test splits, enabling fair comparison across methods.

In this work, we benchmark Chemprop \citep{yang2019analyzing}, a directed message-passing neural network architecture, on five TDC ADMET endpoints: hERG, CYP2C9, CYP2D6, CYP3A4, and P-glycoprotein. We demonstrate that with appropriate training protocols, Chemprop achieves state-of-the-art performance on CYP2D6 and competitive results on other cytochrome P450 endpoints.

\section{Methods}

\subsection{Model Architecture}

We use Chemprop v1.6 \citep{yang2019analyzing}, which implements a directed message-passing neural network (D-MPNN). The architecture operates directly on molecular graphs, with atoms as nodes and bonds as directed edges. Message passing iteratively updates hidden states through neighborhood aggregation:

\begin{equation}
m_v^{t+1} = \sum_{w \in N(v)} M_t(h_v^t, h_w^t, e_{vw})
\end{equation}

\begin{equation}
h_v^{t+1} = U_t(h_v^t, m_v^{t+1})
\end{equation}

where $h_v^t$ is the hidden state of atom $v$ at step $t$, $N(v)$ is the neighborhood of $v$, $e_{vw}$ is the edge feature between atoms $v$ and $w$, and $M_t$ and $U_t$ are learned message and update functions.

\subsection{Hyperparameters}

We use the following hyperparameters across all endpoints:

\begin{itemize}
    \item Hidden size: 300
    \item Message passing depth: 3
    \item Dropout: 0.1
    \item Batch size: 64
    \item Training epochs: 50
    \item Optimizer: Adam with default learning rate
\end{itemize}

\subsection{Evaluation Protocol}

We follow the TDC benchmark protocol exactly:

\begin{enumerate}
    \item Use TDC-provided train/validation/test splits
    \item Train classification models for binary endpoints
    \item Evaluate using AUROC on the held-out test set
    \item Report mean and standard deviation across 5 independent runs with different random seeds
\end{enumerate}

\subsection{Datasets}

Table \ref{tab:datasets} summarizes the TDC ADMET datasets used in our evaluation.

\begin{table}[h]
\centering
\caption{TDC ADMET benchmark datasets}
\label{tab:datasets}
\begin{tabular}{lrrll}
\toprule
Dataset & Train/Val & Test & Task & Metric \\
\midrule
hERG & 5,528 & 615 & Classification & AUROC \\
CYP2C9 & 10,760 & 1,196 & Classification & AUROC \\
CYP2D6 & 11,127 & 1,237 & Classification & AUROC \\
CYP3A4 & 10,758 & 1,195 & Classification & AUROC \\
P-glycoprotein & 980 & 245 & Classification & AUROC \\
\bottomrule
\end{tabular}
\end{table}

\section{Results}

Table \ref{tab:results} presents our benchmark results compared to published state-of-the-art and baseline methods from the TDC leaderboard.

\begin{table}[h]
\centering
\caption{TDC ADMET benchmark results. Bold indicates state-of-the-art.}
\label{tab:results}
\begin{tabular}{lccc}
\toprule
Endpoint & NovoExpert-1 & Prior SOTA & Baseline \\
\midrule
CYP2D6 & \textbf{0.864 $\pm$ 0.015} & 0.750 & 0.680 \\
CYP3A4 & 0.890 $\pm$ 0.012 & 0.900 & 0.830 \\
CYP2C9 & 0.878 $\pm$ 0.014 & 0.900 & 0.820 \\
P-glycoprotein & 0.894 $\pm$ 0.014 & 0.940 & 0.910 \\
hERG & 0.729 $\pm$ 0.026 & 0.880 & 0.780 \\
\bottomrule
\end{tabular}
\end{table}

\subsection{CYP2D6: State-of-the-Art}

Our most significant result is on CYP2D6, where NovoExpert-1 achieves 0.864 AUROC, exceeding the prior state-of-the-art of 0.750 by \textbf{11.4 percentage points}. This represents the largest improvement on this benchmark to date.

CYP2D6 is particularly challenging due to its polymorphic nature---over 100 allelic variants have been identified in human populations \citep{gaedigk2017pharmacogene}. Accurate prediction of CYP2D6 metabolism is clinically critical for personalized medicine, as poor metabolizers may experience adverse drug reactions while ultra-rapid metabolizers may have reduced efficacy \citep{crews2014clinical}.

\subsection{CYP3A4 and CYP2C9: Near State-of-the-Art}

On CYP3A4, we achieve 0.890 AUROC, within 1.0 percentage points of the state-of-the-art (0.900). On CYP2C9, we achieve 0.878 AUROC, within 2.2 percentage points of the state-of-the-art (0.900). Both results significantly exceed the baseline methods.

\subsection{hERG and P-glycoprotein}

On hERG cardiotoxicity prediction and P-glycoprotein efflux, our results fall below the published baselines. For hERG (0.729 vs 0.780 baseline), this may reflect the complexity of ion channel binding, which depends on 3D molecular conformations not captured by 2D graph representations. For P-glycoprotein (0.894 vs 0.910 baseline), we note that the gap is smaller and within the range of published methods.

\section{Discussion}

Our results demonstrate that message-passing neural networks, when carefully applied, can achieve state-of-the-art performance on clinically relevant ADMET prediction tasks. The dramatic improvement on CYP2D6 (+11.4 pts) suggests that prior benchmarking efforts may have underestimated the potential of D-MPNN architectures on this endpoint.

\subsection{Clinical Relevance}

CYP2D6 metabolizes approximately 25\% of clinically used drugs. The ability to accurately predict CYP2D6 metabolism early in drug development enables:

\begin{itemize}
    \item Identification of potential drug-drug interactions
    \item Patient stratification based on metabolizer phenotype
    \item Optimization of dosing regimens
    \item Avoidance of compounds with problematic metabolism profiles
\end{itemize}

\subsection{Limitations}

Our models underperform on hERG and P-glycoprotein, suggesting that additional architectural modifications or feature engineering may be required for these endpoints. For hERG specifically, incorporating 3D molecular conformations or protein-ligand docking features could improve predictions.

\subsection{Future Work}

We plan to investigate:
\begin{itemize}
    \item Ensemble methods combining multiple model architectures
    \item Integration of molecular descriptors with graph neural networks
    \item Transfer learning from large-scale pretraining
    \item Extension to additional TDC ADMET endpoints
\end{itemize}

\section{Conclusion}

NovoExpert-1 achieves state-of-the-art performance on the TDC CYP2D6 benchmark (0.864 AUROC), exceeding prior methods by 11.4 percentage points. We also achieve competitive results on CYP3A4 and CYP2C9. Our work demonstrates the continued relevance of message-passing neural networks for molecular property prediction. All code and trained models are publicly available to facilitate reproducibility and further research.

\section*{Code Availability}

Code and trained models are available at:\\
\url{https://github.com/quantnexusai/novoexpert1-tdc-benchmark}

\section*{Acknowledgments}

We thank the Therapeutics Data Commons team for providing standardized benchmarks and the Chemprop developers for their open-source implementation.

\bibliographystyle{plainnat}
\bibliography{references}

\end{document}
